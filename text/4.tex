\problem{5}
The Boltzmann distribution can be derived from 
the entropy of a closed system.
The fundamental ansatz of statistical mechanics
is that the probability of observing a macrostate is
proportional to the number of corresponding microstates, i.e.:
\[ p_j^{(\rm universe)} \propto e^{S_j^{\rm (universe)}/\kB} \]
We use the subscript $j$ to indicate the entropy (number of microstates)
corresponding to the energy level $E_j$. 
Let the universe be a closed system consisting of two subsystems,
the system of interest and a surrounding reservoir.
The two subsystems are in thermal equilibrium.
Using the condition above, 
show that the probability of observing the system of interest
in a state with energy $E_j$ is proportional to:
\[ p^{(\rm system)}_j \propto e^{-(E_j^{\rm (sys)} - T S_j^{\rm (sys)}) / (\kB T)}
                            = e^{- F_j^{\rm (sys)} / (\kT)} , \]
{\bf Hint:} You may find it useful to work 
relative to an arbitrary reference state $E_0$. 
Note that $e^{S_j - S_0} \propto e^{S_j}$.
\solution{ \\
As stated in the hint, we can write the entropy of the universe relative to some
relative arbitrary state $S_0$:
\[ p_j^{\rm (universe)} \propto e^{S_j^{\rm (universe)}/\kB} 
                        \propto e^{\Delta S_j^{\rm (universe)}/\kB} \]
Setting a reference just scales the exponential by a constant factor,
which gets canceled out by normalization (i.e. the partition function).
\[ e^{\Delta S_j^{\rm (universe)}} = e^{S_j^{\rm (universe)}-S_0}
  = e^{-S_0} e^{S_j^{\rm (universe)}} = C e^{S_j^{\rm (universe)}} \]
Divide the entropy into a system term and reservoir term
(Implicitly, $S_0 = S_0^{\rm (universe)} = S_0^{\rm (sys)} + S_0^{\rm (res)}$). 
\[ e^{\Delta S_j^{\rm (universe)}/\kB} 
 = e^{\Delta S_j^{\rm (sys)}/\kB + \Delta S_j^{\rm (res)}/\kB} \]
As usual, we will assume the reservoir is large enough such that
$\Delta S_j^{\rm (res)} = \Delta E_j^{\rm (res)}/T$.
\[ e^{\Delta S_j^{\rm (sys)}/\kB + \Delta S_j^{\rm (res)}/\kB} 
 = e^{\Delta S_j^{\rm (sys)}/\kB + \Delta E_j^{\rm (res)}/(\kT)} \]
The universe, of course, is closed (constant $E$). 
Therefore, $\Delta E_j^{\rm (res)} = -\Delta E_j^{\rm (sys)}$.
\[ e^{\Delta S_j^{\rm (sys)}/\kB + \Delta E_j^{\rm (res)}/(\kT)}
 = e^{\Delta S_j^{\rm (sys)}/\kB - \Delta E_j^{\rm (sys)}/(\kT)} 
 \propto e^{-(E_j^{\rm (sys)} - T S_j^{\rm (sys)})/(\kT)} \]
This is precisely what we wanted to show.
We have shown, with the maximum entropy condition, that
\[ \boxed{ p^{\rm (sys)}_j = e^{- F_j^{\rm (sys)} / (\kT)} } \]
\newpage
}


\bigskip
\problem{15}
Consider a system of $N$ particles coupled
to external reservoir such that the pressure~$p$ 
and temperature~$T$ are held constant.
We would like to determine the probability of different
macrostates $(N,p,T)$.

\smallskip \subp
Energy and volume both fluctuate in this system.
Find the probability distribution function $p_j$
using the approach discussed in class.
\solution{\\
We would like to maximize entropy 
($S = -\kB \sum_i p_i \ln p_i$)
subject to the following three constraints 
using the Lagrange multipliers
$\lambda_0$, $\lambda_1$, and $\lambda_2$.
\begin{align*}
 \lambda_0 ~ &: ~ \sum_i p_i = 1 \\
 \lambda_1 ~ &: ~ \sum_i p_i E_i = U  \\
 \lambda_2 ~ &: ~ \sum_i p_i V_i = V 
\end{align*}
Here, $U$ and $V$ are constants
constraining the average energy and volume.
In the following, subscripted $p_j$ denotes 
probability, and $p$ without a subscript 
denotes pressure (sorry!).
\begin{align*}
    \frac{\partial}{\partial p_j} 
    &\left[ S - \lambda_0 \left( \sum_j p_j - 1 \right) 
             - \lambda_1 \left( \sum_j p_j E_j - U \right) 
             - \lambda_2 \left( \sum_j p_j V_j - V \right) \right] \\ \\
  = &\left[ -\kB (1 + \ln p_j) - \lambda_0
           -\lambda_1 E_j - \lambda_2 V_j \right] 
  = 0 \\
\end{align*} 
Rearrange the expression to get:
\[ p_j = \exp \left( - \frac{\lambda_0 - \kB 
       + \lambda_1 E_j + \lambda_2 V_j}{\kB} \right) 
= \exp \left( 1 - \frac{\lambda_0}{\kB} \right) 
  \exp \left( -\frac{\lambda_1 E_j + \lambda_2 V_j}{\kB} \right) \]
To assign values to $\lambda_1$ and $\lambda_2$, 
note $S = -\kB \sum_i p_i \ln p_i = -\kB \ave{\ln p_i}$:
\[ S = -\kB \ave{  (1 - \lambda_0/\kB ) 
     - \frac{\lambda_1 E_j + \lambda_2 V_j}{\kB} } 
     = ({\rm constant}) + \lambda_1 \ave{E} + \lambda_2 \ave{V} \]
We can interpret $\ave{E}$ and $\ave{V}$ as the
classical thermodynamic $U$ and $V$:
\[ \dd S = \lambda_1\,\dd U + \lambda_2\,\dd V \]
Given that $\dd U = T\,\dd S - p\,\dd V$,
It is clear that $\lambda_1 = 1/T$ and
$\lambda_2 = p/T$.
Finally, eliminate $\lambda_0$ by 
forcing probability to sum to $1$.
Define the partition function:
\[ \Delta \equiv \sum_i 
   \exp \left( \frac{-E_j - p V_j}{\kT} \right) \]
Set $\lambda_0$ via the relation
$\exp \left( 1 - \frac{\lambda_0}{\kB} \right) \equiv \frac{1}{\Delta}$.
We arrive at the final result:
\[ \boxed{ p_j = \frac{1}{\Delta} 
   \exp \left( \frac{- E_j - p V_j}{\kT} \right) } \]
\[ \boxed{ \Delta \equiv \sum_i 
   \exp \left( \frac{-E_j - p V_j}{\kT} \right) } \]
}

\smallskip \subp
Denote the partition function as $\Delta(N,p,T)$.
We would like you to 
(i) identify the thermodynamic potential
corresponding to $\ln \Delta(N,p,T)$, 
(ii) find an expression for $U = \ave{E}$,
and (iii) show that your expression for $U$ is consistent with
Euler's equation, $U - TS + pV - \mu N = 0$.
\solution{\\
(i) As discussed in class/class notes:
\[ \boxed{ G = -\kT \ln \Delta(N,p,T) } \]
(ii) Begin with $\Delta \equiv \sum_i \exp \left( \frac{-E_i - p V_i}{\kT} \right)$.
Use $\beta \equiv \frac{1}{\kT}$ and $p$ as variables:
\begin{align*}
    \pdc{\ln \Delta}{\beta}{p} &= \frac{1}{\Delta} \pdc{\Delta}{\beta}{p} 
  = \frac{1}{\Delta} \left[ \sum_i (-E_i - p V_i) 
                    \exp \left( \frac{-E_i - p V_i}{\kT} \right) \right] \\
 &= -\ave{E} - p \ave{V} \\
    \pdc{\ln \Delta}{p}{\beta} &= \frac{1}{\Delta} \pdc{\Delta}{p}{\beta} 
  = \frac{1}{\Delta} \left[ \sum_i \frac{-V_i}{\kT} 
                    \exp \left( \frac{-E_i - p V_i}{\kT} \right) \right] \\
 &= - \beta \ave{V}
\end{align*}
Combining these expressions (and correcting signs and factors of $\beta$):
\[ \boxed{ U = \ave{E} = -\pdc{\ln \Delta}{\beta}{p}
                  + p \kT \pdc{\ln \Delta}{p}{\beta} } \]
(iii) Let's convert the derivatives in the $\ave{E}$ equation
into classical thermodynamic quantities:
\begin{align*}
  -\pdc{\ln \Delta}{\beta}{p} &= \pdc{(\beta G)}{\beta}{p}
  = \beta \pdc{G}{T}{p} \pdc{T}{(1/\kT)}{p} + G \pdc{\beta}{\beta}{p} \\
 &= \beta (-S) (-\kT^2) + G = TS + G \\
    p \kT \pdc{\ln \Delta}{p}{\beta} &= -p \kT \pdc{(\beta G)}{p}{\beta} 
  = -p \pdc{G}{p}{T} = -pV
\end{align*}
Substituting into the expression from (ii):
\[ \ave{E} = TS + G - pV \]
Rearrange and we recover the definition of the Gibbs' Free Energy:
\[ \boxed{ G \equiv U - TS + pV }\]
\kh{I'm confident this is all you can do. 
I don't see how you can pull $\mu$ or $N$ out of 
the $(N,p,T)$ ensemble. You could do the following,
but it is equivalent to the previous statement.}
We have shown, purely from manipulating the partition function,
that $\ave{E} = TS + G - pV$.
Now, given that $G \equiv U - TS + pV$, 
Euler's equation clearly reads $G = \mu N$.
Thus:
\[ \boxed{\ave{E} = TS + \mu N - p V } \]
}

\smallskip \subp
Calculate the magnitude of the following fluctuations:
$ \ave{(\delta U)^2} $,
$ \ave{(\delta V)^2} $,
and $ \ave{ \delta U \delta V } $.
Express your answers in terms of the canonical material properties,
$c_{\rm P}$, $\alpha$, and $\kappa_{\rm T}$.
\solution{\\
In order to best relate these quantities
to classical thermodynamics, 
choose $(\beta \equiv 1/\kT, p)$ as working variables.
Since variances are typically given by
partials of expectation values, we evaluate
the following three derivatives as a starting point.
Note $\ave{V}$ is simpler than $\ave{E}$ 
with our choice of variables, so we
take the mixed derivative with respect to $\ave{V}$. \newpage
\begin{align*}
    \pdc{\ave{V}}{p}{\beta}
 &= \pd{}{p} \left[ \sum_i p_i V_i \right]_\beta
  = \pd{}{p} \left[ \frac{1}{\Delta} \sum_i V_i 
                    \exp \left( \frac{-E_j - p V_j}{\kT} 
                                         \right) \right]_\beta \\
 &= \frac{\Delta}{\Delta^2} 
    \sum_i \left( \frac{-V_i^2}{\kT} \right) 
    \exp \left( \frac{-E_j - p V_j}{\kT} \right)
  - \left( \frac{-1}{\kT \Delta^2} \right) \left[ \sum_i V_i 
    \exp \left( \frac{-E_j - p V_j}{\kT} \right) \right]^2 \\
 &= - \frac{\ave{V^2}}{\kT}
    + \frac{\ave{V}^2}{\kT}
  = \frac{-1}{\kT} \ave{(\delta V)^2}
\end{align*}
\begin{align*}
    \pdc{\ave{V}}{\beta}{p}
 &= \pd{}{\beta} \left[ \frac{1}{\Delta} \sum_i V_i 
                        \exp \left( \frac{-E_j - p V_j}{\kT} 
                                             \right) \right]_p \\
 &= \frac{\Delta}{\Delta^2} 
    \sum_i \left( V_i (-E_j - p V_j) \right) 
    \exp \left( \frac{-E_j - p V_j}{\kT} \right)
  - \frac{1}{\Delta^2} \left[ \sum_i V_i 
    e^{\left( \ldots \right)} \right] 
    \left[ \sum_i (-E_j - p V_j)
    e^{\left( \ldots \right)} \right] \\
 &= -\ave{EV} - p\ave{V^2} + \ave{E}\ave{V} + p\ave{V}^2
  = -\ave{\delta E \delta V} - p\ave{(\delta V)^2}
\end{align*}
\begin{align*}
    \pdc{\ave{E}}{\beta}{p}
 &= \pd{}{\beta} \left[ \frac{1}{\Delta} \sum_i E_i 
                        \exp \left( \frac{-E_j - p V_j}{\kT} 
                                             \right) \right]_p \\
 &= \frac{\Delta}{\Delta^2} 
    \sum_i \left( E_i (-E_j - p V_j) \right) 
    \exp \left( \frac{-E_j - p V_j}{\kT} \right)
  - \frac{1}{\Delta^2} \left[ \sum_i E_i 
    e^{\left( \ldots \right)} \right] 
    \left[ \sum_i (-E_j - p V_j)
    e^{\left( \ldots \right)} \right] \\
 &= -\ave{E^2} - p\ave{E V} + \ave{E}^2 + p\ave{E}\ave{V}
  = -\ave{(\delta E)^2} -p \ave{\delta E \delta V}
\end{align*}
Combining the three derivatives above, you can see that:
\begin{align*}
  \ave{(\delta V)^2} &= 
      -\kT \pdc{\ave{V}}{p}{\beta} \\
  \ave{\delta E \delta V} &= 
      -\pdc{\ave{V}}{\beta}{p} + p\kT \pdc{\ave{V}}{p}{\beta} \\
  \ave{(\delta E)^2} &= 
      -\pdc{\ave{E}}{\beta}{p} + p \pdc{\ave{V}}{\beta}{p} 
     - p^2 \kT \pdc{V}{p}{\beta} \\
\end{align*}
To connect these quantities to classical thermodynamics,
let $U \equiv \ave{E}$, $V \equiv \ave{V}$, and note
that constant $\beta$ is the same as constant $T$. 
We can use the chain rule to simplify derivatives in terms of $\beta$:
\[ \pd{}{\beta} = \pd{}{T} \pd{T}{\beta} 
 = \pd{}{T} \left( \pd{\beta}{T} \right)^{-1} = -\kT^2 \pd{}{T} \]
\begin{align*} 
  \ave{(\delta V)^2} &= 
  - \kT \pdc{\ave{V}}{p}{\beta} 
  = -\kT \pdc{V}{p}{T} = \kT V \kappa_T \\ \\
  \ave{\delta E \delta V} &= 
  - \pdc{\ave{V}}{\beta}{p} + p\kT \pdc{\ave{V}}{p}{T} 
  = - (-\kT^2) \pdc{V}{T}{p} - \kT pV \kappa_T \\ 
 &= \kT^2 V \alpha - \kT pV \kappa_T 
  = \kT V ( \alpha T - p \kappa_T) \\ \\
  \ave{(\delta E)^2} &= 
  - \pdc{\ave{E}}{\beta}{p} + p \pdc{\ave{V}}{\beta}{p} 
  - p^2 \kT \pdc{V}{p}{T} \\
 &= -(-\kT^2) \pdc{U}{T}{p} - \kT^2 p V \alpha + \kT p^2 V \kappa_T \\
 &= \kT^2 (C_p- p V \alpha) - \kT^2 p V \alpha + \kT p^2 V \kappa_T \\
 &= \kT^2 C_p + \kT pV (p \kappa_T - 2 \alpha T)
\end{align*}
The final answer is therefore:
\[ \boxed{ \ave{(\delta V)^2} 
         = \kT V \kappa_T  } \]
\[ \boxed{ \ave{\delta E \delta V}
         =  \kT V ( \alpha T - p \kappa_T)  } \]
\[ \boxed{ \ave{(\delta E)^2} 
         = \kT^2 C_p + \kT pV (p \kappa_T - 2 \alpha T) } \]
One nice check you can do with these three results:
\[ \ave{(\delta H)^2} = \ave{(\delta U + p \delta V)^2} 
 = \ave{(\delta E)^2} + 2 p \ave{\delta E \delta V} 
 + p^2 \ave{(\delta V)^2} = \kT^2 C_p \]
\newpage
}


\bigskip
\problem{15}
Liquid water typically contains dissolved salts.
We would like to investigate how dissolved ions
affect solution thermodynamics.
To aid in our analysis, assume the solution is incompressible
and divide the solution volume $V$ into
$N$ equal cells of volume $v$.
To keep things simple, assume that solvent molecules, cations, and anions 
each occupy one cell volume ($v = 30\,\text{\AA}^3$),
and that the solution is at constant temperature $T$. 
Let $n_1$ denote the number of solvent molecules,
and $n_2 = N - n_1$ denote the total number of ions.

\smallskip \subp
Let us begin by looking at the free energy of an ``ideal'' salt solution. 
Assume that ions do not interact and treat cations and anions as the same species.
(i) Write down an expression for $\Omega$,
the number of ways solvent and solute molecules 
can be distributed in the $N$ cells, and
(ii) using your expression for $\Omega$, 
show that the free energy can be written
\[
F_0 = -TS = - k_{\rm B} T \ln \Omega =
N k_{\rm B} T \left[ \phi \ln(\phi) + (1 - \phi) \ln(1 - \phi) \right]
\]
Here, $\phi \equiv n_2/N$ is the mole fraction of ions. 
\solution{ \\
$\Omega$ comes from a simple combinatorial expression. If ions are the same:
\[ \boxed{ \Omega = \binom{N}{n_1} = \frac{N!}{n_1!(N-n_1)!}  
                                   = \frac{N!}{n_1!\,n_2!} } \]
Apply Stirling's approximation to simplify $\ln \Omega$.
\begin{align*}
\ln \Omega &= \ln N! - \ln n_1! - \ln n_2! \\ 
   &= \ln (n_1 + n_2)! - \ln n_1! - \ln n_2! \\
   &\approx (n_1 + n_2) \ln (n_1 + n_2) - n_1 \ln n_1 - n_2 \ln n_2 
           + n_1 + n_2 - n_1 - n_2 \\ 
   &= n_1 \ln \left( \frac{n_1 + n_2}{n_1} \right) 
    + n_2 \ln \left( \frac{n_1 + n_2}{n_2} \right) \\
   &= -n_1 \ln \left( \frac{n_1}{N} \right) - n_2 \ln \left( \frac{n_2}{N} \right)
\end{align*}
Given that $\phi \equiv n_2/N$ and $N \equiv n_1 + n_2$,
it is clear that $1 - \phi \equiv n_1/N$. 
\begin{align*}
   F_0 &= -\kT \ln \Omega = -\kT \left[ -n_1 \ln \left( \frac{n_1}{N} \right) 
         - n_2 \ln \left( \frac{n_2}{N} \right) \right] \\
       &= N \kT \left[ (1 - \phi) \ln (1 - \phi) + \phi \ln (\phi) \right]
\end{align*}
\[ \boxed{ F_0 = N \kT \left[ \phi \ln (\phi) + (1-\phi) \ln (1-\phi) \right] } \]
}


\smallskip \subp
In reality, salt solutions are non-ideal because
ions exhibit long-ranged Coulombic interactions.
As a result, cations in solution 
are typically surrounded by a `cloud' of anions, and vice versa. 
The size of this ion `cloud' is characterized 
by the Debye length, $\lambda_{\rm D}$
($V_{\rm cloud} \simeq \lambda_{\rm D}^3$).
\\ \\
Let us focus on a single ion.
On average, this ion is surrounded by $n_{\rm cloud}$ ions
of the opposite charge.
Each ion in this `cloud' contributes a Coulombic energy
of order $1 / \lambda_{\rm D}$. 
The total energy felt by the ion is of order $\kT$,
i.e. $n_{\rm cloud} \times (\kappa / \lambda_{\rm D}) \simeq k_{\rm B} T$,
where $\kappa$ is a positive constant.
Using this assertion, show that
\[ \lambda_{\rm D} \propto \phi^{-1/2} \]
\solution{  
The average number of ions in the cloud is given by the bulk concentration:
\[ n_{\rm cloud} = c_{\rm ion} V_{\rm cloud}
 = \left(\frac{\phi}{v}\right) \lambda_{\rm D}^3 \] 
\[ n_{\rm cloud} \left( \frac{\kappa}{\lambda_{\rm D}} \right)
 = \frac{\phi \lambda_{\rm D}^3}{v} 
   \left( \frac{\kappa}{\lambda_{\rm D}} \right)
 = \frac{\kappa \phi \lambda_{\rm D}^2}{v} \simeq \kT \]
We have assumed that the net electrostatic 
energy of an ion at equilibrium is of order $\kT$. 
Rearrange the expression to get:
\[ \boxed{ \lambda_{\rm D} \simeq \left( \frac{\kT v}{\kappa \phi} \right)^{1/2} 
           \propto \phi^{-1/2} }\]
}


\smallskip \subp
The result from part (b) can be used to show
that the free energy of the non-ideal solution is:
\begin{equation}
F = F_0 - k_{\rm B} T \frac{V}{\lambda_{\rm D}^3}
= N k_{\rm B} T \left( f_0 - \alpha \phi^{3/2} \right).
\label{eq:fVO}
\end{equation}
Here, $f_0 \equiv F_0 / (N k_{\rm B} T)$ and
$\alpha$ is a positive constant that doesn't depend on $\phi$.
The $\phi^{3/2}$ term is negative because
Coulombic interactions favor ion aggregation and
destabilize mixing. \\ \\ 
Starting from the expression above, find 
(i) the chemical potential of the solvent ($\mu_1(\phi)$),
(ii) the chemical potential of the ions ($\mu_2(\phi)$),
and (iii) show that these results are consistent
with the Gibbs-Duhem relation for solutions, 
$ (1-\phi)\,\dd \mu_1 + \phi\,\dd \mu_2 = 0 $.
\solution{ \\
Define $f \equiv F/N$. Since $f(\phi)$ depends only on $\phi$:
\begin{align*}
\mu_1 &= \left. \pd{(N f(\phi))}{n_1} \right|_{n_2,T}
       = N \pd{f}{\phi} \pdc{\phi}{n_1}{n_2} + f \pdc{N}{n_1}{n_2}
       = N \pd{f}{\phi} \left( \frac{-\phi}{N} \right) + f(\phi) \\
      &= f(\phi) - \phi f'(\phi) \\
\mu_2 &= \left. \pd{(N f(\phi))}{n_2} \right|_{n_1,T} 
       = N \pd{f}{\phi} \pdc{\phi}{n_2}{n_1} + f \pdc{N}{n_2}{n_1}
       = N \pd{f}{\phi} \left( \frac{1-\phi}{N} \right) + f(\phi) \\
      &= f(\phi) + (1-\phi) f'(\phi)
\end{align*}
As defined, we can write:
\[ f(\phi)  = \kT \left[ \phi \ln (\phi) + (1-\phi) \ln (1-\phi) 
            - \alpha \phi^{3/2} \right] \]
\[ f'(\phi) = \kT \left[ \ln (\phi) - \ln (1-\phi) 
            - \frac{3 \alpha}{2} \phi^{1/2} \right] \]
Applying the above, it is straightforward to show that:
\[ \boxed{ \text{(i) } \mu_1 = \kT \left[ \ln (1-\phi) 
                + \frac{\alpha \phi^{3/2}}{2} \right]} \]
\[ \boxed{ \text{(ii) } \mu_2 = \kT \left[ \ln \phi 
                - \frac{3 \alpha \phi^{1/2}}{2}
                + \frac{\alpha \phi^{3/2}}{2} \right]} \] \\
(iii) To show Gibbs-Duhem (at constant $T$),
wrap the $\kT$ factor into the expression to simplify it
($\mu/\kT \equiv \beta \mu$ drops constant $\kT$ factor):
$(1-\phi)\,\dd (\beta \mu_1) + \phi\,\dd (\beta \mu_2) = 0$.
\[ (1-\phi)\,\dd(\beta \mu_1) =
   (1-\phi) \left[ \frac{-1}{1-\phi} 
 + \frac{3 \alpha \phi^{1/2}}{4} \right]\,\dd \phi 
 = \left[ -1 + \frac{3 \alpha \phi^{1/2}}{4} 
             - \frac{3 \alpha \phi^{3/2}}{4} \right]\,\dd \phi \]
\[ \phi\,\dd(\beta \mu_2) = 
   \phi \left[ \frac{1}{\phi} - \frac{3 \alpha}{4 \phi^{1/2}} 
             + \frac{3 \alpha \phi^{1/2}}{4} \right]\,\dd \phi 
 = \left[ 1 - \frac{3 \alpha \phi^{1/2}}{4} 
            + \frac{3 \alpha \phi^{3/2}}{4} \right]\,\dd \phi \]
Clearly, $(1-\phi)\,\dd \mu_1 = -\phi\,\dd \mu_2$.
Thus, we have shown:
\[ \boxed{(1-\phi)\,\dd \mu_1 + \phi\,\dd \mu_2 = 0} \] \\ \\
{\bf Alternative Solution for $\mu_1$ and $\mu_2$: } 
If you prefer to use $n_1$ and $n_2$ directly:
\begin{align*}
F &= N \kT \left[ \phi \ln (\phi) + (1-\phi) \ln (1-\phi) 
     - \alpha \phi^{3/2} \right] \\
  &= \kT \left[ n_2 \ln \left( \frac{n_2}{n_1 + n_2} \right) 
              + n_1 \ln \left( \frac{n_1}{n_1 + n_2} \right) 
              - \alpha \left( \frac{n_2^3}{n_1 + n_2} \right)^{1/2} \right]
\end{align*}
Since $\dd F = -S\,\dd T + \mu_1\,\dd n_1 + \mu_2\,\dd n_2$
\begin{align*}
  \frac{\mu_1}{\kT} 
        &= \frac{1}{\kT} \pdc{F}{n_1}{n_2,T} \\
        &= \frac{-n_2}{n_1 + n_2} 
         + \frac{n_2}{n_1 + n_2} 
         + \ln \left( \frac{n_1}{n_1 + n_2} \right) 
         + \frac{\alpha}{2} \left( \frac{n_2}{n_1+n_2} \right)^{3/2}  \\
        &= \ln (1 - \phi) + \alpha \frac{\phi^{3/2}}{2} \\
  \frac{\mu_2}{\kT} 
        &= \frac{1}{\kT} \pdc{F}{n_2}{n_1,T} \\ 
        &= \frac{n_1}{n_1 + n_2} 
         + \ln \left( \frac{n_2}{n_1 + n_2} \right)
         - \frac{n_1}{n_1 + n_2} 
         - \frac{3\alpha}{2} \left( \frac{n_2}{n_1 + n_2} \right)^{1/2}
         + \frac{\alpha}{2} \left( \frac{n_2}{n_1 + n_2} \right)^{3/2} \\
        &= \ln \phi - \frac{3 \alpha \phi^{1/2}}{2} + \frac{\alpha \phi^{3/2}}{2}
\end{align*}
\newpage
}




\bigskip
\problem{15}
Perhaps surprisingly, it is rather painful to get saltwater in your nose.
This pain is caused by the osmotic pressure of dissolved ions,
which results from composition fluctuations. 
Let us determine this osmotic pressure.
The results from {\bf Problem~3} will be needed
to complete some of these questions.
Once again, assume that saltwater is incompressible, 
and subdivide the solution volume $V$ into $N$ identically-sized 
cells with volume $v$. 
the $n_1$ solvent molecules and $n_2 = N - n_1$ ions,
have the same volume $v$.  
We define $\phi \equiv n_2/N$. 

\smallskip \subp
Although the solution is macroscopically homogeneous,
there are microscopic fluctuations because
the contents of each `cell' fluctuate.
Show that, in the absence of Coulombic interactions,
composition fluctuations in a single cell are given by
($\phi_0 \equiv \ave{\phi}$ is the bulk mole fraction):
\begin{equation}
\ave{(\delta \phi)^2} = \phi_0 (1 - \phi_0) ,
\label{eq:random}
\end{equation}
Since only one particle may occupy a cell at a time,
it is useful to think of $\phi_0 = \ave{\phi} = \ave{n_{\rm ion}}$,
where $n_{\rm ion} = \{0\text{ or }1\}$ is the number of ions in the cell.
\solution{ \\
Let $n_{\rm ion}$ be the number
of ions in a single cell (either zero or one).
We are given that $\phi_0 = \ave{n_{\rm ion}}$.
Since this is a two state system, $\phi_0$ 
defines the probability of seeing an ion in the cell.
\[ \ave{n_{\rm ion}} = \sum_i p_i n_i 
 = (1-\phi_0)*(0) + \phi_0*(1) = \phi_0 \]
We can directly evaluate:
\[ \ave{n_{\rm ion}^2} = \sum_i p_i n_i^2
= (1-\phi_0)*(0)^2 + \phi_0*(1)^2 = \phi_0 \]
Thus we have shown that $\ave{\phi_0} = \ave{\phi_0^2}$. 
Evaluate the variance: 
\[ \ave{(\delta \phi)^2} = \ave{\phi^2} - \ave{\phi}^2 
 = \ave{n_{\rm ion}^2} - \ave{n_{\rm ion}}^2
 = \phi_0 - \phi_0^2 = \phi_0 (1 - \phi_0) \]
\[ \boxed{ \ave{(\delta \phi)^2} = \phi_0 (1 - \phi_0) } \] 
}

\smallskip \subp
More effort is required if Coulombic interactions
cannot be neglected.
The free energy per cell is $F/N \equiv f(\phi)$,
where $F$ is given by Eq.~(\ref{eq:fVO}).
Let us view each cell as a subsystem 
capable of mass exchange with its surroundings
via the ``exchange'' chemical potential
$w = \frac{\partial f}{\partial \phi} \equiv f'(\phi_0)$.
The resulting probability of observing a cell with composition $\phi$ is:
\[ p(\phi) \propto e^{- (f(\phi) + w \phi)/(k_{\rm B} T) } \]
This defines the generating/partition function 
$Z(w) \equiv \int\!\dd\phi\,e^{- (f(\phi) + w \phi)/(k_{\rm B} T) }$.
It follows that:
\[ \ave{(\delta \phi)^2} = (k_{\rm B}T)^2 \frac{\partial^2 \ln Z(w)}{\partial w^2}
= k_{\rm B} T \frac{\partial \ave{\phi}}{\partial w}
= k_{\rm B} T \frac{\partial \phi_0}{\partial w} \]
This is analogous to the evaluation of $\ave{(\delta N)^2}$ from
the grand canonical ensemble.
Using $w \equiv f'(\phi_0)$, the last expression can be rearranged to yield
(for a single volume cell):
\[ \ave{(\delta \phi)^2} = \frac{k_{\rm B} T}{f''(\phi_0)} \]
(i) Show that as $\alpha \rightarrow 0$,
this expression is consistent with Eq.~(\ref{eq:random}).
(ii) Write an expression for the total composition fluctuation
in $m$ cells, then
(iii) write an expression for the thermodynamic stability limit
of the salt solution. What happens to $\ave{(\delta \phi)^2}$
at this limit? Why?
\solution{ \\
\[ f(\phi) = \kT \left[\phi \ln (\phi) + (1-\phi) \ln (1-\phi)
           - \alpha \phi^{3/2} \right] \]
Since $f(\phi)$ is a function of only $\phi$, 
$f''(\phi)$ is just the second derivative with respect to $\phi$.
\begin{align*}
f''(\phi) &= \frac{\dd}{\dd \phi} \left[ \frac{\dd f(\phi)}{\dd \phi} \right]
           = \kT \frac{\dd}{\dd \phi} \left[
             \ln (\phi) - \ln (1-\phi) 
           - \frac{3 \alpha}{2} \phi^{1/2} \right] \\
          &= \kT \left[ \frac{1}{\phi} + \frac{1}{1-\phi}
                      - \frac{3 \alpha}{4} \phi^{-1/2} \right] 
           = \kT \left[ \frac{1}{\phi(1-\phi)} 
                      - \frac{3 \alpha}{4} \phi^{-1/2} \right]
\end{align*}
(i.) Take the limit as $\alpha \rightarrow 0$:
\[ \boxed{ \lim_{\alpha \rightarrow 0} \ave{(\delta \phi)^2} 
         = \frac{k_{\rm B} T}{f''(\phi_0)}
         = \frac{\kT}{\kT} \left( \frac{1}{\phi_0(1-\phi_0)} \right)^{-1}
         = \phi_0 (1-\phi_0) } \]
This is a good indication our work in 4(a) was correct! \\ \\
(ii.) The total free energy of $m$ cells is
$F = m f(\phi)$. For simplicity, we just assume
that when $m$ is very large, the cells are uncorrelated.
Though this assumption is annoying, consider
that only cells that are close to each other should have
correlated fluctuations.
In the limit that $m \rightarrow \infty$, 
these local correlations become unimportant.
Thus, multiply $f(\phi)$ by $m$ to get:
\[ \boxed{ \ave{(\delta \phi_{\rm total})^2} 
         = \frac{\kT}{m f''(\phi_0)} 
         = \frac{1}{m} \ave{(\delta \phi)^2} } \] \\
(iii.) We are interested in the stability condition
with respect to composition fluctuations. 
These are governed by fluctuations in the 
extensive variables $n_1$ and $n_2$.
In Homework 2, you showed that stability conditions from
simultaneous fluctuations in extensive variables were
equivalent to the conditions obtained by considering
fluctuations in one variable only. 
We take the converse - stability is equivalently 
governed by simultaneous fluctuation in $n_1$ and $n_2$.
The critical mode (i.e. composition change fluctuation)
corresponds to $\dd n_1 = - \dd n_2$. 
Recognizing this is enough justification to use $\phi$,
but if you want, it's not too bad to show mathematically:
\begin{align*}
    \pdc{^2 F}{n_1^2}{n_2,T} &= \pdc{^2 (Nf(\phi))}{n_1^2}{n_2,T}
  = \pd{}{n_1} \left[ N \pdc{f}{n_1}{n_2} + f \pdc{N}{n_1}{n_2} \right]_{n_2,T} \\
 &= \pdc{\phi}{n_1}{n_2} \pd{}{\phi}
    \left[N \pdc{\phi}{n_1}{n_2} \pd{f}{\phi} + f \right] \\
 &= \left( \frac{-\phi}{N} \right) \pd{}{\phi}
    \left[-\phi \pd{f}{\phi} + f \right] \\
 &= \left( \frac{-\phi}{N} \right)
    \left[-\phi \pd{^2 f}{\phi^2} - \pd{f}{\phi} + \pd{f}{\phi} \right] 
  = \left( \frac{\phi^2}{N} \right) \pd{^2 f}{\phi^2}
\end{align*}
Remember that we {\bf minimize} free energies. 
The thermodynamic stability limit is:
\[ \boxed{ \left( \frac{\partial^2 f}{\partial \phi^2} \right)_T
 = f''(\phi) > 0 } \]
The ideal term ($\phi(1-\phi)$, where $0 \leq \phi \leq 1$) 
is always positive, so we approach the stability limit 
from the positive side ($f''(\phi) \rightarrow 0^+$)
when $\alpha$ becomes large. 
At the stability limit, since the composition fluctuation
is proportional to $1/f''(\phi)$, 
$\ave{(\delta \phi)^2} \rightarrow \infty$. 
\[ \boxed{ \ave{(\delta \phi)^2} \text{ diverges at the 
   stability limit due to phase transition.} }\]
A phase transition represents macroscopic segregation/ordering,
which can only occur when microscopic fluctuations become
macroscopically in scale. \\
}

\smallskip \subp
The osmotic pressure $\Pi$ pushes solvent molecules
from regions of low to high solute concentration.
It accounts for the free energy change per cell
after a solvent molecule is displaced.
\[ \Pi = - \frac{1}{v} \left( \frac{\partial F}{\partial n_1}  \right)_{n_2} 
       = - \frac{\mu_1}{v} \]
As seen above, osmotic pressure is directly related to the chemical potential.
Write an expression for the osmotic pressure
in the limit that the solution is very dilute. 
Use the non-ideal form of chemical potential 
you obtained in {\bf Problem~3(c)}.
Keep only the leading-order term.
\solution{ \\
Plugging in the expression for $\mu_1$ from Problem 3(c):
\[ \Pi = -\frac{\mu_1}{v} 
       = \frac{\kT}{v} 
         \left[ \ln (1-\phi) + 
         \frac{\alpha \phi^{3/2}}{2} \right] \]
In the limit that the solution is dilute, $\phi \rightarrow 0$.
We can Taylor expand the log term to get:
\[ \lim_{\phi \rightarrow 0} \Pi \approx \frac{-\kT}{v}
   \left[ -\phi - \frac{\phi^2}{2} - \cdots 
          + \frac{\alpha \phi^{3/2}}{2} \right] \approx
   \frac{\kT \phi}{v} \]
Notice that the leading order term 
does not depend on $\alpha$!
\[ \boxed{ \lim_{\phi \rightarrow 0} \Pi \approx 
           \frac{\kT \phi}{v} } \]
}

\smallskip \subp
Saltwater typically has a concentration of
$\phi_0 / v = \rm 0.3\,nm^{-3}$.
Calculate the room temperature osmotic pressure 
of saltwater using your result from part~(c).
Use ``bar'' as your units of pressure.
Is it surprising that this pressure is painful?
\solution{\\
\[ \Pi \approx (1.38*10^{-23}\,{\rm J/K})*(300\,{\rm K})
              *(0.3*10^{27}\,{\rm m}^{-3}) 
             = 12.4\,{\rm bar} \]
This is quite a large osmotic pressure! 
Our estimate from (c) is an upper bound, 
because we have neglected the opposing electrostatic term
(that depends on $\alpha$), but nevertheless, 
the osmotic pressure is on the order of $10$ atm.
Not a big surprise that it hurts!
\[ \boxed{ \Pi \approx 12.4\,{\rm bar} } \]
}

\iffalse
\smallskip \subp
The free energy expression we arrived at implies that,
when solution is sufficiently concentrated or
temperature is sufficiently low,
the originally homogeneous solution can undergo a phase-separation,
resulting in a water-rich supernatant
coexisting with an ion-rich concentrated electrolyte solution.
The nature of this phase separation is analogous to
the liquid-vapor coexistence predicted by the Van der Waals
equation of state:
The conjugate pair $(\Pi, \phi)$ for solution is fully analogous
to the pair $(p, V)$ for Van der Waals fluid.
Drawing from this analogy, find out the
compositions for the two coexisting phases, i.e.,
the binodal compositions, for several values of $\alpha$.
Find out the values of $\alpha$ and $\phi$ at the critical point.
Plot the values you obtained in the $\alpha$-$\phi$ plane.
{\sl FYI: This is the simplest model for studying the
phase separation in aqueous solutions, coacervation.}
\fi

\iffalse % Keep in case students make same mistake
\begin{align*}
   \dd \left( \frac{\mu_1}{\kT} \right) &= 
       \pdc{(\beta \mu_1)}{n_1}{n_2,T} \dd n_1 
     + \pdc{(\beta \mu_1)}{n_2}{n_1,T} \dd n_2 \\
    &= \left[ \frac{n_2}{N n_1} + \frac{3}{4 n_2} 
       \left( \frac{n_2}{N} \right)^{5/2} \right]\,\dd n_1
     + \left[-\frac{1}{N} - \frac{3 n_1}{4 (N)^2} 
              \left( \frac{n_2}{N} \right)^{1/2} \right]\,\dd n_2
\end{align*}
\begin{align*}
   \dd \left( \frac{\mu_2}{\kT} \right) &= 
       \pdc{(\beta \mu_2)}{n_1}{n_2,T} \dd n_1 
     + \pdc{(\beta \mu_2)}{n_2}{n_1,T} \dd n_2 \\
    &= \left[-\frac{1}{N} - \frac{3}{4 n_2} 
              \left( \frac{n_2}{N} \right)^{3/2} 
            + \frac{3}{4 n_2} 
              \left( \frac{n_2}{N} \right)^{5/2} \right]\,\dd n_1
     + \left[ \frac{n_1}{N n_2} + \frac{3 n_1}{4}
              \left( \frac{n_2}{N} \right)^{1/2} 
            - \frac{3 n_1}{4N^2} 
              \left( \frac{n_2}{N} \right)^{1/2} \right]\,\dd n_2
\end{align*}
\fi