%%%%%%%%%% Problem 1 %%%%%%%%%%
\problem{10}
Two equations of state have been proposed for rubber bands 
(based on L.~R.G. Treloar's work, 1943):
\begin{align}
S &= L_0 \gamma(\theta U/L_0)^{1/2} - L_0\gamma\left[
\frac{1}{2}\left(\frac{L}{L_0}\right)^2 + \frac{L_0}{L} - \frac{3}{2}\right] \\
S &= L_0 \gamma\ee^{\theta n U/L_0} - L_0\gamma\left[
\frac{1}{2}\left(\frac{L}{L_0}\right)^2 + \frac{L_0}{L} - \frac{3}{2}\right]
\end{align}
Above, $\gamma$, $l_0$, and $\theta$ are constants, 
$L$ is the variable length of the rubber band,
$L_0 = n l_0$ is the rest length,
and $S$ and $U$ have their usual meaning.
Here, $n$ is a `mass-like' quantity
describing the size of the band.

\smallskip
\subp Which of the two equations is thermodynamically acceptable? Why? 
\solution{ \\
We need to check if entropy is extensive.
By definition, $U$ must be extensive.
Both $n$ and $L$ must be extensive because
the rest length and extended length
should scale with the size of the rubber band.
Therefore, we are trying to show that 
$S(\lambda U,\lambda n,\lambda L) = \lambda S(U,n,L)$. \\ \\
Equation (1):
\begin{align*}
S(\lambda U, \lambda n, \lambda L) &= 
       \lambda L_0 \gamma (\theta \lambda U / (\lambda L_0))^{1/2} 
     - \lambda L_0 \gamma \left[
       \frac{1}{2} \left( \frac{\lambda L}{\lambda L_0} \right)^2 
     + \frac{\lambda L_0}{\lambda L} - \frac{3}{2}\right ] \\
    &= \lambda L_0 \gamma \left[ (\theta U/L_0)^{1/2} 
     - \frac{1}{2} \left( \frac{L}{L_0} \right)^2 - \frac{L}{L_0}
     + \frac{3}{2} \right] \\
    &= \lambda S(U, n, L)
\end{align*}
Equation (2):
\begin{align*}
S(\lambda U, \lambda n, \lambda L) &= 
       \lambda L_0 \gamma \exp\left( \frac{\theta \lambda n \lambda U}
                                          {\lambda L_0} \right) 
     - \lambda L_0 \gamma \left [ \frac{1}{2} \left( 
       \frac{\lambda L}{\lambda L_0} \right ) ^2 
     + \frac{\lambda L_0}{\lambda L} - \frac{3}{2} \right] \\
    &= \lambda L_0 \gamma \left[ \exp\left( \frac{\textcolor{red}{\lambda} \theta n U}
                                                 {L_0} \right) 
     - \frac{1}{2} \left( \frac{L}{L_0} \right)^2 - \frac{L_0}{L}
     + \frac{3}{2} \right] \\
    &\neq \lambda S(U,V,n)
\end{align*}
The second equation is invalid because entropy isn't extensive.
\begin{center}
\fbox{Only equation (1) is thermodynamically acceptable.}
\end{center}
}

\smallskip
\subp
For the acceptable choice, (i) find the expression for temperature $T$,
and (ii) deduce the dependence of the tension $f$ upon $T$ and $L/L_0$ --
that is, find $f(T,L/L_0)$.
Here $f$ is defined as the force required to 
stretch the rubber band to length $L$.
{\bf Hint}: {\sl The first law is $\dd U = T \dd S + f \dd L$\/}. 
\solution{ \\
Given the hint, it follows that $T = \pdc{U}{S}{n,L}$. By the inverse rule:
\[ \frac{1}{T} = \pdc{S}{U}{n,L} 
 = \frac{L_0 \gamma}{2} \left( \frac{\theta}{U L_0} \right)^{1/2} \] 
\[ \boxed{\text{(i) } T = \frac{2}{\gamma}
          \left( \frac{U}{L_0 \theta} \right)^{1/2}} \]
The other partial derivative of $\dd U = T \dd S + f \dd L$ yields $f$: 
\[ f = \pdc{U}{L}{S} = -\pdc{S}{L}{U} \pdc{U}{S}{L} 
     = -T \pdc{S}{L}{U}
     = T L_0 \gamma \left( \frac{L}{L_0^2} - \frac{L_0}{L^2} \right)  
     = \gamma T \left( \frac{L}{L_0} - \frac{L_0^2}{L^2} \right) \]
\[ \boxed{ \text{(ii) } f(T,\frac{L}{L_0}) = 
   \gamma T \left( \frac{L}{L_0} - \frac{L_0^2}{L^2} \right) } \]
}




%%%%%%%%%% Problem 2 %%%%%%%%%%
\bigskip\problem{15}
The Helmholtz free energy for some fluid is given by:
\[ F(N,V,T) = Nf_0 - Ns_0(T-T_0) - \frac{a N^2}{V} 
            - NRT \ln \left(\frac{V/N-b}{v_0} \right) 
            + N \int_{T_0}^T \left( \frac{T'-T}{T'} \right)
              c_V^\text{ideal}(T')\,\dd T'\, . \]
Above, $f_0$, $s_0$, $T_0$, $a$, and $b$ are constants.
All other symbols adopt their standard meanings.

\smallskip\subp Find the internal energy expressed in terms of $N$, $V$ and $T$.
\solution{\\
By definition, $F = U - TS$. 
Therefore, to get $U(N,V,T)$, we need to find $S(N,V,T)$:
\begin{align*}
 S(N,V,T) &= -\left (\frac{\partial F}{\partial T} \right )_{V,N} \\
          &= Ns_0 + NR \ln \left( \frac{V/N -b}{v_0} \right) 
              - N \frac{\partial}{\partial T} \int_{T_0}^T 
              \left( \frac{T^{\prime} - T}{T^{\prime}} \right)
              c_v ^{\mathrm{ideal}}(T^{\prime})\,dT^{\prime} \\
          &= Ns_0 + NR \ln \left( \frac{V/N -b}{v_0} \right) 
              - N \frac{\partial}{\partial T} \int_{T_0}^T 
                c_v ^{\mathrm{ideal}}(T^{\prime})\,dT^{\prime} 
              + N \frac{\partial}{\partial T} \int_{T_0}^T 
                \frac{T}{T^{\prime}}c_v^{\mathrm{ideal}}(T^{\prime})\,dT^{\prime} \\
          &= Ns_0 + NR \ln \left( \frac{V/N -b}{v_0} \right)
              + N \int_{T_0}^T 
                \frac{c_v^{\mathrm{ideal}} (T^{\prime})}{T^{\prime}}\,dT^{\prime} 
\end{align*}
Comparing the expressions for $F(N,V,T)$ and $TS(N,V,T)$, applying $U = F + TS$ yields
\[ \boxed{ U(N,V,T) = N f_0 + N T_0 s_0 - \frac{aN^2}{V} 
                    + N \int_{T_0}^T c_v^{\mathrm{ideal}} (T^{\prime{}})dT^{\prime} } \]
}



\smallskip\subp Find the mechanical equation of state, i.e., $PVT$ relationship, for this fluid.
\solution{\\
We need an expression for $p$. 
Since $\dd F = -S \dd T - p \dd V + \mu \dd N$:
\[ p = -\pdc{F}{V}{N,T} = \frac{aN^2}{V^2} + NRT\frac{v_0}{V/N - b}\frac{1}{N v_0} \]
\[ \boxed{ p = -\frac{aN^2}{V^2} + \frac{NRT}{V - Nb} }\]
You should recognize this as the Van der Waals equation of state.
\newpage{}
}



\smallskip\subp Find the isothermal compressibility
$\kappa_T \equiv -\frac{1}{V}\pdc{V}{P}{T}$
and the (isobaric) thermal expansion coefficient $\alpha \equiv \frac{1}{V} \pdc{V}{T}{P}$.
\solution{\\
The Van der Waals equation is cubic in $V$; 
for simplicity, only take derivatives of $P$.
\[ \pdc{p}{V}{T} = \frac{\partial}{\partial V} \left( \frac{NRT}{V - Nb}-\frac{aN^2}{V^2} \right)_T 
                 = \frac{-NRT}{(V - Nb)^2} + \frac{2 aN^2}{V^3}  \]
\[ \pdc{p}{T}{V} = \frac{\partial}{\partial T} \left( \frac{NRT}{V - Nb}-\frac{aN^2}{V^2} \right)_V
                 = \frac{NR}{V-Nb} \]
The quantity $\kappa_T$ can be evaluated directly:
\[ \pdc{V}{p}{T} = \pdc{p}{V}{T}^{-1} 
                 = \frac{1}{\frac{-NRT}{(V - Nb)^2} + \frac{2 aN^2}{V^3} }
                 = \frac{V^3 (V-Nb)^2}{-NRTV^3 + 2aN^2 (V-Nb)^2} \]
\[ \boxed{ \kappa_T = \frac{V^2 (V-Nb)^2}{NRTV^3 - 2aN^2 (V-Nb)^2} } \]
For $\alpha$, we can get rid of the annoying constant $p$ with the triple product rule:
\[ \pdc{V}{T}{p} = - \pdc{p}{T}{V} \pdc{V}{p}{T} \]
\[ \alpha = \frac{1}{V} \pdc{V}{T}{p} 
          = -\frac{1}{V} \left( \frac{NR}{V - Nb} \right)
                         \left( \frac{V^3 (V-Nb)^2}{-NRTV^3 + 2aN^2 (V-Nb)^2} \right) \]
\[ \boxed{ \alpha = \frac{NR V^2 (V-Nb)}{NRTV^3 - 2aN^2 (V-Nb)^2} } \]
}




%%%%%%%%%% Problem 3 %%%%%%%%%%
%\newpage
\bigskip
\problem{15}
Though we typically ignore interfacial effects in bulk systems,
surface energy cannot be neglected in systems of modest size.
We can quantify the excess surface free energy
using the surface tension $\gamma$, which opposes increases in surface area.

\smallskip\subp
Treat the surface area $A$ as an additional extensive variable.
Write down (i) the generalized differential form of internal energy
with surface contributions, (ii)
the corresponding form of the Helmholtz free energy,
and (iii) the Gibbs-Duhem relation.
\solution{ \\
The surface energy term is positive because 
increasing area will increase energy.
\[ \boxed{ \text{(i) }\dd U = T\,\dd S - p\,\dd V + \mu\,\dd N + \gamma\,\dd A} \]
The Legendre transform defines $F \equiv U - TS$. Naturally,
\[ \boxed{\text{(ii) }\dd F = -S\,\dd T - p\,\dd V + \mu\,\dd N + \gamma\,\dd A} \]
We have added the extensive-intensive variable pair $(A,\gamma)$.
Deriving Euler's equation as we did in class gives
$U = TS - pV + \mu N + \gamma A$. 
Comparing the total differential of this expression to
$\dd U$ as given in (i) yields the Gibbs-Duhem relation:
\[ \boxed{\text{(iii) } S\,\dd T - V\,\dd p + N\,\dd \mu + A\,\dd \gamma = 0} \]
}

\smallskip\subp
The entropy is now also a function of surface area, $A$.
Write down the thermodynamic stability condition 
corresponding to fluctuations in only the surface area.
\solution{\\
The stability condition requires that the entropy of a closed system
is maximized with respect to fluctuations in the 
extensive variables $(N,V,U,A)$. 
\[\boxed{ \left.\frac{\partial^2 S(N,V,U,A)}{\partial A^2}\right|_{N,V,U} 
        = -\pdc{(\gamma/T)}{A}{N.V,U} < 0 } \] 
Note that one can also derive the stability condition by looking
at the curvature of the free energy {\bf with respect to an extensive variable}.
(Be careful here, e.g. the curvature of $F$ with respect to $T$,
an intensive quantity, doesn't tell you anything!)
\[\boxed{ \left.\frac{\partial^2 F(N,V,T,A)}{\partial A^2}\right|_{N,V,T} 
        = \pdc{\gamma}{A}{N.V,T} > 0 } \] 
}

\smallskip\subp
Estimate the order of magnitude of the interfacial tension
between water and air at room temperature,
and compare your estimate to the known value $\rm72\,mN/m$. 
\solution{\\
(Any reasonable solution within an order of magnitude was accepted) \\ \\
Assume that in liquid water, on average, each water molecule
forms four hydrogen bonds (tetrahedral geometry) 
with energy $\approx-10\,\kT$.
This yields $\approx-20\,\kT$ of hydrogen bond energy
per molecule in the bulk ($4*10/2$, division corrects for double-counting).
Water molecules at the interface cannot form hydrogen bonds with air;
on average, these molecules should have $1/6$ less hydrogen
bonds than those in the bulk
(There are six directions corresponding to
$(\pm x, \pm y, \pm z)$, and we are losing say, $+z$). 
Therefore, the energy gain per water molecule at the surface is
about $20/6 \approx 3.3\,\kT/\text{molecule}$. \\ \\
Now we must estimate the area density of water molecules at the surface.
Begin by finding the average distance between molecules 
using the average volume per molecule:
\[ \frac{1000\,\text{kg}}{\text{m}^3} \times 
   \frac{1\,\text{m}^3}{10^{30}\,\text{\AA}^3} \times
   \frac{1\,\text{mol}}{0.018\,\text{kg}} \times
   \frac{6.022*10^{23}}{1\,\text{mol}} 
   \approx 0.033\,\text{\AA}^{-3} \]
\[ \left( \frac{1\,\text{\AA}^3}{0.033} \right)^{1/3} 
   \approx 3\,\text{\AA} \]
Therefore, the average surface area per molecule at the
interface is $\approx 9\,\text{\AA}^2$. Thus:
\[ \frac{3.3\,\kT}{\text{molecule}} \times
   \frac{\text{molecule}}{9\,\text{\AA}^2} =
   \frac{3.3*1.38*10^{-23}*300\,\text{J}}{9*10^{-20}\,\text{m}^2} =
   0.15\,\text{J}/\text{m}^2 \]
\[ \boxed{\gamma \approx 150\,\text{mN}/\text{m} } \]
This value is pretty close!
In reality, there are less than four hydrogen bonds per liquid water molecule
(forming all possible hydrogen bonds makes ice).
Our estimate is an upper bound.
Interfacial tension also depends on entropic effects
that were not treated in our approximation.
}

\smallskip\subp
Consider a spherical droplet of radius $R$ immersed in
a continuum media with pressure $P_0$.
Minimize the Helmholtz free energy with respect to variations
in only the radius $R$ (constant $T$ and $N$),
and write down the equilibrium condition.
\solution{ \\
Remember that $\dd F = -S\,\dd T - p\,\dd V + \mu\,\dd N + \gamma\,\dd A$.
Both $\dd V$ and $\dd A$ are coupled to $R$.
\begin{align*}
V = \frac{4\pi}{3} R^3 \qquad &\longrightarrow \qquad \dd V = 4\pi R^2 \dd R \\
A = 4\pi R^2 \qquad &\longrightarrow \qquad \dd A = 8\pi R \dd R
\end{align*}
Assuming constant $N$ and $T$, we find the minimum of $F$
by setting $\pdc{F}{R}{T,N} = 0$:
\[ \pdc{F}{R}{T,N} = -p \pdc{V}{R}{T,N} + \gamma \pdc{A}{R}{T,N} 
                   = -p (4\pi R^2) + \gamma (8\pi R) = 0 \]
Thus, the equilibrium condition is:
\[ \boxed{p \equiv P_\text{bubble} - P_0  = \frac{2\gamma}{R}} \]
Note that $p$ is the pressure differential 
between the inside and outside of the bubble.
}

\smallskip\subp 
Using the equilibrium condition you obtained in part (d),
calculate the pressure inside a spherical micro-bubble of air
with radius $\rm5\,\mu m$ surrounded by water at pressure $P_0=1\,\rm bar$.
\solution{\\
\[ p \equiv P_\text{bubble} - P_0 = \frac{2 \gamma}{R} = 
     \frac{2*0.072\,\text{N}/\text{m}}{5*10^{-6}\,\text{m}} \approx 
      29\,\text{kPa} \]
To check if $p$ is indeed positive,
remember that $P_\text{bubble}$ must be larger than $P_0$ or else
surface tension will drive the bubble to collapse ($A \rightarrow 0$). 
\[ \boxed{P_\text{bubble} = 100\,\text{kPa} + 29\,\text{kPa} 
                    \approx 129\,\text{kPa}} \]
}

%%%%%%%%%% Problem 4 %%%%%%%%%%
%\bigskip
%\rightline{({\sl Problem 4 on back of page})}
%\vfil\eject
%\medskip
\bigskip\problem{20}
In class, we derived stability conditions for a closed system
with respect to fluctuations in 
energy $U$, volume $V$, and the number of particles $N$.
Though we only considered one fluctuation at a time, 
multiple extensive quantities may fluctuate simultaneously.
Consider the case where both energy and volume can fluctuate,
but the number of particles is held constant (no $\delta N$ fluctuation).

\smallskip\subp Find the $2\times2$ array for the second order differentials
${\bf J} = \left.\frac{\partial^2 S(U,V)}{\partial(U,V)^2}\right|_N $.
\solution{\\
\begin{align*}
\bf{J} 	&= \begin{bmatrix} 
        \left(\frac{\partial^2S}{\partial V^2}\right)_{U,N} & 
      	\frac{\partial}{\partial U}	\left(\left(\frac{\partial S}{\partial V}\right)_{U,N}\right)_{V,N} \\
        \frac{\partial}{\partial V} \left(\left(\frac{\partial S}{\partial U}\right)_{V,N}\right)_{U,N} &
        \left(\frac{\partial^2S}{\partial U^2}\right)_{V,N}
        \end{bmatrix}
\end{align*}
}

\smallskip\subp Express each entry of ${\bf J}$ in terms of
the three canonical measurable properties, $\alpha$, $\kappa_T$, and $C_p$.
\solution{\\
All derivatives have constant $N$, 
so $N$ will be omitted from the constant variables for simplicity. 
Start with the differential form of entropy as a function of $U$ and $V$ 
(from the first law):
\begin{align*}
  \dd S &= \left.\frac{\partial S}{\partial U}\right|_V \,\dd U
         + \left.\frac{\partial S}{\partial V}\right|_U \,\dd V \\
	&= \frac{1}{T}\,\dd U+\frac{p}{T}\,\dd V
\end{align*}
We can now fill in the first derivatives for the 
four elements in the $\bf{J}$ matrix:
\[ \bf{J} = \begin{bmatrix} 
        \frac{\partial}{\partial V }\left(\frac{p}{T}\right)_{U} & 
        \frac{\partial}{\partial U}\left(\frac{p}{T}\right)_{V} \\
        \frac{\partial}{\partial V}\left(\frac{1}{T}\right)_{U} &
        \frac{\partial}{\partial U }\left(\frac{1}{T}\right)_{V}
        \end{bmatrix} \]
We will also find it helpful to relate the constant volume heat capacity ($C_V$)
and constant pressure heat capacity ($C_p$). 
This is not necessary to solve this problem 
(especially if the thermo-partial solver is used), 
but it will simplify the math greatly:
\[C_p = T\pdc{S}{T}{p} \quad C_v 
      = T\pdc{S}{T}{V} \quad \mathrm{(by\ definition)} \]
\[dS = \pdc{S}{p}{T}dp+\pdc{S}{T}{p}dT \]
\[dS = -\pdc{V}{T}{p}dp+\frac{C_p}{T}dT\]
\[dS = -\alpha V dp+\frac{C_p}{T}dT \]
\[C_v = T\pdc{S}{T}{V} = T \left[-\alpha V \pdc{p}{T}{V}+\frac{C_p}{T}\pdc{T}{T}{V} \right] = T \left[\alpha V \frac{\pdc{V}{T}{p}}{\pdc{V}{p}{T}}+\frac{C_p}{T} \right] \]
\[C_v =  -\frac{\alpha^2 VT}{\kappa_T}+C_p = \frac{C_p\kappa_T-\alpha^2VT}{\kappa_T} \]
We must now calculate the second derivative for each entry:
\begin{align*}
	\left(\frac{\partial^2S}{\partial U^2}\right)_{V} = \frac{\partial}{\partial U }\left(\frac{p}{T}\right)_{V} 	
	&= -\frac{1}{T^2}\pdc{T}{U}{V} = -\frac{1}{T^2C_V} = -\frac{\kappa_T}{T^2(C_p\kappa_T-\alpha^2VT)}
\end{align*}
From Schwarz theorem, we know that the order of differentiation doesn't matter for analytical functions, thus the cross terms are equal:
\begin{align*} 
  \pdc{\pdc{S}{V}{U}}{U}{V} 
  &= \pdc{\pdc{S}{U}{V}}{V}{U} \\
  &= \frac{\partial}{\partial V}\left(\frac{1}{T}\right)_{U} \\
  &= -\frac{1}{T^2}\pdc{T}{V}{U} \\
  &= \frac{1}{T^2}\frac{\pdc{U}{V}{T}}{\pdc{U}{T}{V}} \\
  &= \frac{1}{T^2C_V}\pdc{U}{V}{T} = \frac{1}{T^2C_V}\left(T\pdc{S}{V}{T}-p\pdc{V}{V}{T} \right) \\
  &= \frac{1}{T^2C_V}\left(T\pdc{p}{T}{V}-p \right) = \frac{1}{T^2C_V}\left(\frac{\alpha T}{\kappa_T}-p \right) \\ 
  \pdc{\pdc{S}{V}{U}}{U}{V} &= \frac{1}{T^2}\frac{\alpha T-p\kappa_T}{(C_p\kappa_T-\alpha^2 VT)}
\end{align*}
The final derivative to be calculated is $\left(\frac{\partial^2S}{\partial V^2}\right)_{U,N}$:
\begin{align*}
	\left(\frac{\partial \frac{p}{T}}{\partial V }\right)_{U} 
   	&= -\frac{p}{T^2}\pdc{T}{V}{U}+\frac{1}{T}\pdc{p}{V}{U}\\
\end{align*}
We know $\pdc{T}{V}{U}$ from our solution to the cross-differentials. Just need to calculate $\pdc{p}{V}{U}$:
\begin{align*}
	\pdc{p}{V}{U}
    &= -\frac{\pdc{U}{V}{p}}{\pdc{U}{p}{V}}=-\frac{(T\pdc{S}{V}{p}-p)}{(T\pdc{S}{p}{V}-p\pdc{V}{p}{V})} \\
    &= \frac{(T\pdc{P}{T}{S}-p)}{T\pdc{V}{T}{S}} = -\frac{(T\frac{-\pdc{S}{T}{p}}{\pdc{S}{p}{T}}-p)}{T\frac{\pdc{S}{T}{V}}{\pdc{S}{V}{T}}} \\
    &= -\frac{(\frac{C_p}{\pdc{V}{T}{p}}-p)}{\frac{C_V}{\pdc{P}{T}{V}}} = -\frac{(\frac{C_p}{\alpha V}-p)}{\frac{C_V\kappa_T}{\alpha}} \\
    &= -\frac{(C_p-p\alpha V)}{C_V \kappa_T V}
\end{align*}
\begin{align*}
	\left(\frac{\partial \frac{p}{T}}{\partial V }\right)_{U}
    &= -\frac{p}{T^2}\frac{\left(p\kappa_T-\alpha T\right)}{C_V \kappa_T}-\frac{1}{T}\frac{(C_p-p\alpha V)}{C_V \kappa_T V} \\
    &= -\frac{p^2V\kappa_T - pV\alpha T +C_pT-pV\alpha T}{C_V\kappa_T T^2 V} \\
    &= -\frac{p^2V\kappa_T - pV\alpha T +C_pT-pV\alpha T}{C_V\kappa_T T^2 V} \\
    &= -\frac{p^2V\kappa_T - 2pV\alpha T +T(C_V +\frac{\alpha ^2TV}{\kappa_T})}{C_V\kappa_T T^2 V} \\
    &= -\frac{T C_V + \frac{V}{\kappa_T}(p^2\kappa_T^2 - 2p\alpha T\kappa_T +\alpha^2T^2)}{C_V\kappa_T T^2 V} \\
    &= -\frac{T C_V \kappa_T + V(p\kappa_T - \alpha T)^2}{C_V\kappa_T^2 T^2 V}
    = -\frac{T (\kappa_T C_p - \alpha^2VT) + V(p\kappa_T - \alpha T)^2}{(\kappa_T C_p - \alpha^2VT)\kappa_T T^2 V}
\end{align*}
Note that in the above set of equations, we used the identity relating $C_p$ and $C_V$ to greatly simplify the math. Now we can plug all the answers back into the original matrix:
\renewcommand\arraystretch{3}
\[\boxed{\bf{J} 	= 
		\begin{bmatrix} 
          -\dfrac{T (\kappa_T C_p - \alpha^2VT) + V(p\kappa_T - \alpha T)^2}{(\kappa_T C_p - \alpha^2VT)\kappa_T T^2 V} & 
          \dfrac{1}{T^2}\dfrac{\alpha T-p\kappa_T}{(C_p\kappa_T-\alpha^2 VT)} \\
          \dfrac{1}{T^2}\dfrac{\alpha T-p\kappa_T}{(C_p\kappa_T-\alpha^2 VT)} &
          -\dfrac{\kappa_T}{T^2(C_p\kappa_T-\alpha^2VT)}
        \end{bmatrix}}
\]
\renewcommand\arraystretch{1} 
\newpage }


\smallskip\subp The stability condition demands that 
the array $\bf J$ is negative-definite (negative curvature).
A $2\times 2$ matrix is negative-definite if both of the following
conditions are satisfied:
\[ {\bf Tr}[{\bf J}] < 0 \qquad \text{and} \qquad {\bf Det}[{\bf J}] > 0 \]
Here, {\bf Tr} and {\bf Det} are the 
matrix trace and determinant, respectively.
Find the generalized stability condition 
when both $U$ and $V$ can fluctuate.
\solution{
\\
We will find the determinant using $C_V$ rather than $C_p$ to simplify the math:
\begin{align*}
\renewcommand\arraystretch{3}
\bf{Det[J]} 	
		&>0 \\
       \bf{Det}		
        \begin{bmatrix} 
          -\dfrac{T C_V \kappa_T + V(p\kappa_T - \alpha T)^2}{C_V\kappa_T^2 T^2 V} & 
          \dfrac{1}{T^2}\dfrac{\alpha T-p\kappa_T}{C_V \kappa_T} \\
          \dfrac{1}{T^2}\dfrac{\alpha T-p\kappa_T}{C_V \kappa_T} &
          -\dfrac{1}{T^2C_V}
        \end{bmatrix} &> 0 \\
        \frac{T C_V \kappa_T + V(p\kappa_T - \alpha T)^2}{C_V^2\kappa_T^2 T^4 V}-\frac{(p\kappa_T-\alpha T)^2}{C_V^2 \kappa_T^2T^4} &> 0 \\
        \frac{T C_V \kappa_T + V(p\kappa_T - \alpha T)^2-V(p\kappa_T-\alpha T)^2}{C_V^2\kappa_T^2 T^4 V} &> 0 \\
        \frac{T C_V \kappa_T}{C_V^2\kappa_T^2 T^4 V} &> 0 \\
        \frac{1}{C_V\kappa_T T^3 V} &> 0 \\
        \mathrm{(Since\ T\ \&\ V\ >\ 0), } \quad \boxed{C_V\kappa_T > 0} \\
        \boxed{C_p\kappa_T >\alpha^2TV}
\end{align*}
Now, let's take a look at the trace. 
\begin{align*}
-\frac{1}{C_V T^2} - \frac{C_V \kappa_T T + V(p\kappa_T - \alpha T)^2}
                          {C_V \kappa_T^2 T^2 V} < 0 \\
 \frac{\kappa_T V + C_V T + \kappa_T^{-1}V(p\kappa_T-\alpha T)^2}
      {C_V \kappa_T T^2 V} > 0 
\end{align*}
The determinant requires that $C_V \kappa_T > 0$, and for this
system we know that $V$ and $T$ are positive. 
Thus, the denominator will not affect the sign of the expression:
\[ \kappa_T V + C_V T + \kappa_T^{-1}V(p\kappa_T-\alpha T)^2 > 0 \]
The determinant condition also requires that $C_V$ and $\kappa_T$
have the same sign. By inspection, every term in the inequality
must have the same sign (the squared term is always positive).
Thus, the only possible way both conditions can be fulfilled is if:
\[ \boxed{ C_V > 0 \qquad \text{and} \qquad \kappa_T > 0 }\]
Side note: the corner terms in the trace summation 
do not have the same units. Therefore, to be rigorous, 
we should include a factor accounting for the conversion
between the arbitrary choice of $U$ and $V$ units.
Let $A > 0$ be this conversion factor:
\begin{align*}
-\frac{A}{C_V T^2} - \frac{C_V \kappa_T T + V(p\kappa_T - \alpha T)^2}
                          {C_V \kappa_T^2 T^2 V} < 0 \\
A \kappa_T V + C_V T + \kappa_T^{-1}V(p\kappa_T-\alpha T)^2 > 0
\end{align*}
Since unit conversion doesn't change signs, it is evident
that mismatched units do not make a difference for this problem.
The terms in the summation are guaranteed to be the same sign,
regardless of scaling, because of the determinant condition.
}


\iffalse
\bigskip
\problem{5}
%{\sl Le Chatlier's principle}.
We have seen in class that the isobaric heat capacity
$C_p \equiv T\! \left.\frac{\partial S}{\partial T}\right|_p$
is always greater than the isochoric heat capacity
$C_V \equiv T\! \left.\frac{\partial S}{\partial T}\right|_V$.
This is not coincidental --
Pressure ($p$) and volume ($V$) are conjugate variables,
and the stability condition demands that
$\frac{\partial p}{\partial V} < 0$.
It can be shown that the partials with
intensive quantities (e.g. $p$) fixed are always greater than those with
extensive quantities (e.g. $V$) fixed.
Using the shape (concave down) of $S(U,V,N)$,
illustrate graphically why $C_p > C_V$. 
\fi