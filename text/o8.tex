\rightline{
{\bf Source:} A.~B. Pippard,
{\sl Elements of Classical Thermodynamics}}

\bigskip\problem{10}
According to some textbooks,
a knowledge of the Joule-Kelvin coefficient of a gas,
and its specific heat at constant pressure,
as functions of temperature and pressure is sufficient
to enable the equation of state to be determined.
Show that this is not so, and that
further information, e.g., the form of one isotherm, is required.

\bigskip\problem{10}
A saturated vapor is expanded adiabatically;
if $L$ is the latent heat of vaporization,
show that it becomes supersaturated or unsaturated according as whether
$-T \frac{\dd}{\dd T}\!\left(\frac{L}{T}\right)$
is greater or lesser than the specific heat
of the liquid under its own vapor pressure.

\bigskip\problem{10}
A simple liquefier is constructed so that compressed gas
enters at room temperature $T_0$ and a high pressure $p$,
and passed through a heat exchange to a throttle,
where it is expanded to a low pressure;
part condenses and the rest returns through the heat exchanger,
leaving the liquefier at room temperature and pressure.
Show that the fraction of gas liquefied is greatest when
the pressure of the gas entering is adjusted so that
$(p, T_0)$ is a point on the inversion curve of the gas.

\bigskip\problem{10}
A Simon helium liquefier consists essentially of a vessel
into which helium gas is compressed to a high pressure $P$ at $10^\circ{\rm K}$
(above the critical point of helium).
The vessel is then thermally isolated,
and the gas is allowed to escape slowly through a capillary tube
until the pressure within the vessel is $1$~atmosphere,
and the temperature $4.2^\circ{\rm K}$,
the normal boiling-point of helium.
Assume that the thermal isolation is perfect,
that the heat capacity of the vessel is negligible
in comparison with that of the gas,
and that the gas obeys the ideal gas law,
calculate what value of $p$ must be chosen
for the vessel to be entirely filled with liquid.
(Latent heat of liquid helium at $4.2^\circ{\rm K}$ is
$20\,{\rm cal\,mol}^{-1}$;
heat capacity $C_V$ of gaseous helium is
$3\,{\rm cal\,mol}^{-1}{\rm K}^{-1}$.)

\bigskip\problem{10}
Show that in a small Joule expansion of a fluid
having negative expansion coefficient
the pressure changes more than in the corresponding adiabatic expansion,
the ratio of the two changes being $1 - pV\alpha / C_p$.

\bigskip\problem{10}
In a certain compressor gas at room temperature $T_0$
and atmospheric pressure $p_0$ is compressed adiabatically,
and is then passed through water-cooled tubes until eventually
it emerges at pressure $p_1$ and temperature $T_0$.
Find an expression for the work required for this process,
compared with what would be needed for a reversible
isothermal compression leading to the same result,
and show that the ratio is not less than unity.
Discuss also the changes of entropy occurring in the two processes.

\bigskip\problem{10}
An ellipsoid made of a magnetically isotropic substance
is free to rotate about a vertical axis
in a uniform horizontal magnetic field,
with two unequal axes horizontal.
The susceptibility is independent of field strength.

\smallskip\subp
Show both by direct calculation of couples and
by the thermodynamic conditions for equilibrium that
the ellipsoid tends to set itself with the longer horizontal axis
(i.e., that having the smaller demagnetizing coefficient)
parallel to the field, whether the substance is paramagnetic or diagmagnetic.

\smallskip\subp
If the ellipsoid is thermally isolated,
and constructed of a paramagnetic material which obeys Curie's law,
derive an expression for the variation of its temperature
when it is rotated in a constant field,
in terms of the demagnetizing coefficients along the principal axes.
For the sake of simplicity assume that the susceptibility
is much smaller than unity.

\bigskip\problem{10}
For a system consisting of two phases of a substance in equilibrium,
the specific heat at constant volume and the adiabatic compressibility
are related by the equation
$$ \frac{C_V}{\kappa_S} = VT \left(\frac{\dd p}{\dd T}\right)^2,$$
where $\dd p/\dd T$ is the slope of the equilibrium line on the phase diagram.
Show that this result holds if the transition between the phases
is either of the first order of a $\lambda$-transition.

\bigskip\problem{10}
On the basis of the following information, which is partly hypothesis
and partly somewhat simplified experimental data,
calculate the melting pressure of ${}_2{\rm He}^3$ at $0\,{\rm K}$
using the following information:
(1)
Between $0$ and $\rm 10^{-5}\,K$ the specific heat of the solid is very high,
but between $10^{-5}$ and $\rm 1\,K$ it is much less than that of the liquid;
(b)
the specific heat of the liquid is proportional to $T$ below $\rm 1\,K$;
(c)
the expansion coefficient of both phases may be assumed to be zero;
(d)
at $\rm 0.32\,K$ the melting pressure $p_{\rm m}$ is $\rm 29.4\,atm$
and $\dd p_{\rm m}/\dd T = 0$;
at $\rm 0.7\,K$ $p_{\rm m}$ is $\rm 33\,atm$.

\bigskip\problem{10}
According to experimental measurements shown in Fig.~\ref{fig:iron},
$\alpha$-iron transforms into $\gamma$-iron at $906^\circ{\rm C}$
and back to $\alpha$-iron at $1400^\circ{\rm C}$.
Between these temperatures the specific heat of $\gamma$-iron
raises linearly from $0.160$ to $\rm 0.169\,cal\,g^{-1}K^{-1}$.
On the assumption that $\alpha$-iron, if it were stable between
$906$ and $1400^\circ{\rm C}$,
would have a specific heat constant at the value $\rm 0.185\,cal\,g^{-1}K^{-1}$
that it has at both these temperatures,
calculate the latent heat at each transition
and comment on the experimental value for the $906^\circ{\rm C}$ transition,
$\rm 3.86\,cal\,g^{-1}$.

\begin{figure}[hb]
\hfil\includegraphics[width=0.4\textwidth,height=!]{iron.eps}\hfil
\caption{Specific heat of iron.}\label{fig:iron}
\end{figure}

\bigskip\problem{10}
The transition point of the
$\rm S^\alpha \rightarrow S^\beta$
transition is $95.5^\circ{\rm C}$,
and the melting point of $\rm S^\beta$ is $119.3^\circ{\rm C}$,
at atmospheric pressure.
The latent heat of the transition $\rm S^\alpha\rightarrow S^\beta$
is $\rm 2.78\,cal\,g^{-1}$,
and the latent heat of fusion of $\rm S^\beta$ is $\rm 13.2\,cal\,g^{-1}$.
The densities of $\rm S^\alpha$, $\rm S^\beta$, and liquid sulphur
are $2.07$, $1.96$, and $\rm 1.90\,g\,cm^{-3}$, respectively.
Assuming the latent heats and densities to be independent of temperature
and pressure, find the coordinate $(P, T)$ of the triple point of
$\rm S^\alpha$, $\rm S^\beta$, and liquid sulphur.

\bigskip\problem{10}
Discuss qualitatively the effects produced by gravity
in experimental determinations of
(a) the specific heat of a substance exhibiting a second-order
or $\lambda$-transition;
(b) the critical isotherm of a simple fluid.

\bigskip\problem{10}
Show that for a substance obeying van~der~Waals's equation,
the latent heat drops to zero with a vertical tangent
as the critical point is approached,
i.e., $\lim_{T\rightarrow T_\text{c}} (\dd L/\dd T) = -\infty$.
What behavior is to be expected for a real substance such as xenon,
for which the isotherms near $T_\text{c}$ are shown in Fig.~\ref{fig:xenon}?

\begin{figure}[hb]
\hfil\includegraphics[width=0.4\textwidth,height=!]{xenon.eps}\hfil
\caption{Isotherm of xenon near the critical point.
The broken lines mark the region of coexistent phases,
and the dotted line is the critical isotherm according to
the van~der~Waals equation.}\label{fig:xenon}
\end{figure}

\bigskip\problem{10}
The specific heat of a unit volume of a metal
may be approximately represented by the formulae
\begin{align*}
C_\text{s} &= a T^3 \phantom{+ \gamma T}\qquad \text{in the superconducting state},\\
C_\text{n} &= a T^3 + \gamma T \qquad \hskip-5pt\text{in the normal state},
\end{align*}
where $a$, $b$, $\gamma$ are constants.
Show that these formulae lead to the following results.

\smallskip\subp
The transition temperature in zero field, $T_\text{c} = (3\gamma/(a-b))^{1/2}$.

\smallskip\subp
The critical magnetic field $H_\text{c} = H_0(1-t^2)$
where $t=T/T_\text{c}$ and $H_0=T_\text{c}(2\pi\gamma)^{1/2}$.

\smallskip\subp
The difference between the internal energies of the two states,
in zero field, has a maximum when the temperature is $T_\text{c}/3$.

\smallskip\subp
If a magnetic field applied to an isolated superconductor
is increased very slowly to a value above the critical,
the transition to the normal state is accompanied by a cooling of the metal.
Find an expression for the drop in temperature.

\smallskip\subp
If the field is applied suddenly, instead of slowly,
the metal is heated rather than cooled if the strength of the field exceeds
$ H_0 [ (1+3t^2) (1-t^2) ]^{1/2} $.

