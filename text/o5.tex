\problem{5}
We showed in class that the energy fluctuation $\ave{(\delta U)^2}$ in canonical ensemble
is given by $k_{\rm\scriptstyle B}T^2 C_V$.
In the grand canonical ensemble with fixed $(\mu,V,T)$, the number of particles $N$ fluctuates,
and a corresponding fluctuation-response relation holds.
Show that the average and variation are given by
\begin{align}
\ave{N} &= \kT \pdc{\ln\Xi(\mu,V,T)}{\mu}{V,T} \nonumber\\
\ave{(\delta N)^2} &= \kT \pdc{\ave{N}}{\mu}{V,T} \label{eq:Nfluct}
\end{align}
where $\Xi$ is the grand canonical partition function
and $\ave{(\delta N)^2}$ is the variance $\ave{(N - \ave{N})^2}$.

\bigskip
\problem{5}
Which of the following curves is a possible candidate for the
$V$-dependence of entropy $S$? Why?

\smallskip
\centerline{\includegraphics[width=0.30\textwidth,height=!]{figs/SofV}}

\bigskip
\bigskip
\problem{12}
Consider the isothermal-isobaric ensemble with fixed $(N,p,T)$,
appropriate for systems in thermal and mechanical contact with exterior reservoirs.
The system energy and volume can fluctuate.
By following the steps explained in lecture and notes for canonical and
grand canonical ensembles, show that maximizing the entropy
$S = -k_{\scriptscriptstyle \rm B} \sum_i \mathcal{P}_i \ln(\mathcal{P}_i)$
subject to the constraints
\begin{align*}
\sum_i \mathcal{P}_i &= 1 \\
\sum_i \mathcal{P}_i\, E_i &= U \\
\sum_i \mathcal{P}_i\, V_i &= V 
\end{align*}
leads to the following statistical properties:

\smallskip\subp
The probability of observing a particular microstate $i$ with energy $E_i$ and volume $V_i$ is proportional to
$$ \mathcal{P}_i\propto\exp\left(-\frac{E_i + p V_i}{\kT}\right) .$$

\smallskip\subp
The partition function
$\Delta (N,p,T) \equiv \sum_i \exp\left(-\frac{E_i + p V_i}{\kT}\right)$
relates to the Gibbs free energy by
$$ G = - \kT \ln \Delta(N,p,T) .$$

\rightline{\sl see next page for more questions}
\vfil\eject

\ \smallskip\subp
The amplitude of volume fluctuation is governed by the response to pressure change
$$ \ave{(\delta V)^2} = -\kT \pdc{\ave{V}}{p}{N,T} = \kT V \kappa_T.$$
Note that this amplitude diverges at the limit of stability, i.e., when $\partial p/\partial{V} = 0$.

\smallskip\subp
Show that, for ideal gases, the relative fluctuation decreases with system size according to 
$$ \frac{\ave{(\delta V)^2}^{1/2}}{\ave{V}} = \frac{1}{\sqrt{N}} . $$

\bigskip
\problem{15}
Denote the canonical partition function of a single particle by $q=q(V,T)$.
In general, $q$ can be written as a product, $q=V f(T)$.
$f(T)$ has dimension of inverse volume, depends on temperature,
and measures the excitation of the internal degrees of freedoms.
Although the functional form of $f(T)$ is not known,
we can derive the equations of states and the thermodynamic potentials using $q$.
Let's consider a gas of a variable number of {\sl non-interacting} and
{\sl indistinguishable} particles confined to a volume $V$.
The gas is held at constant temperature $T$ and chemical potential $\mu$
by coupling with thermal and chemical reservoirs.

\smallskip\subp
Compute the grand canonical ensemble partition function $\Xi(\mu,V,T)$.
It needs to be expressed in terms of $q$, $T$, and $\mu$.
No explicit reference to $V$ or $f(T)$ is needed.

\smallskip\subp
Use the thermodynamic potential $pV = \kT \ln\Xi(\mu,V,T)$
and the Gibbs-Duhem relation to find the average number of particles $\ave{N}$.
Show that it leads to the ideal gas law
$$ pV = \ave{N} \kT .$$

\smallskip\subp
Show that the equation of state relating particle number to chemical potential is given by
$$ \mu = \kT \ln\left(\frac{\ave{N}}{q}\right) .$$
Use this relation and Eq.~(\ref{eq:Nfluct}) to show that $\ave{(\delta N)^2} = \ave{N}$.

\smallskip\subp
Find the entropy by using the Gibbs-Duhem relation, and show that it can be written as
$$ S = \ave{N} \kB\left[ 1 + \ln \left(\frac{q}{\ave{N}}\right) + \dd{\ln f(T)}{\ln T} \right] .$$
Is this entropy extensive? Explain your reasoning.

\smallskip\subp
Find the two remaining potentials, the internal energy $U$ and the Helmholtz free energy $F$,
and show that they are given by
\begin{align*}
U &= \ave{N} \kT \dd{\ln f(T)}{\ln T} , \\
F &= -\ave{N} \kT \left[1 +  \ln \left(\frac{q}{\ave{N}}\right) \right].
\end{align*}
The Gibbs free energy is $G = \ave{N} \mu$, so we did not ask you to calculate it.
